\ifdefined\ISBLUEPRINT\else
\documentclass[11pt]{amsart}

% Package Being Used:

\usepackage{amsmath}
\usepackage{amssymb}
\usepackage{bm}
\usepackage{graphicx}
\usepackage{psfrag}
\usepackage{color}
\usepackage{hyperref}
\hypersetup{colorlinks=true, linkcolor=blue, citecolor=magenta, urlcolor=wine}
\usepackage{url}
\usepackage{algpseudocode}
\usepackage{fancyhdr}
\usepackage{mathtools}
\usepackage{tikz-cd}
\usepackage{xy}
\input xy
\xyoption{all}
\usepackage{stmaryrd}
\usepackage{calrsfs}

% Paper Format and Geometry:

\voffset=-1.4mm
\oddsidemargin=14pt
\evensidemargin=14pt
\topmargin=26pt
\headheight=9pt     
\textheight=576pt
\textwidth=441pt %441
\parskip=0pt plus 4pt

% Head Labels:

\pagestyle{fancy}
\fancyhf{}
\renewcommand{\headrulewidth}{0pt}
\renewcommand{\footrulewidth}{0pt}
\fancyhead[CE]{\fontsize{9}{11}\selectfont F. GOTTI}
\fancyhead[CO]{\fontsize{9}{11}\selectfont }
\fancyhead[LE,RO]{\thepage}

\setlength{\headheight}{9pt}

% Theorems-like Format and Numbering:

\newtheorem*{maintheorem*}{Main Theorem}
\newtheorem{theorem}{Theorem}%[section]
\newtheorem*{theorem*}{Main Theorem}
\newtheorem{prob}[theorem]{Problem}
\newtheorem{question}[theorem]{Question}
\newtheorem{proposition}[theorem]{Proposition}
\newtheorem{conj}[theorem]{Conjecture}
\newtheorem{lemma}[theorem]{Lemma}
\newtheorem{cor}[theorem]{Corollary}

\theoremstyle{definition}
\newtheorem{definition}[theorem]{Definition}
\newtheorem{remark}[theorem]{Remark}
\newtheorem{example}[theorem]{Example}
\numberwithin{equation}{section}
\newtheorem{exe}{Exercise}


% Personalized Commands:
\newcommand{\cc}{\mathbb{C}}
\newcommand{\ff}{\mathbb{F}}
\newcommand{\nn}{\mathbb{N}}
\newcommand{\pp}{\mathbb{P}}
\newcommand{\qq}{\mathbb{Q}}
\newcommand{\rr}{\mathbb{R}}
\newcommand{\zz}{\mathbb{Z}}
\newcommand{\mca}{\mathcal{A}}
\newcommand{\mce}{\mathcal{E}}
\newcommand{\mcr}{\mathcal{R}}
\newcommand{\aaa}{\mathbf{A}}

\newcommand{\lcm}{\text{lcm}}
\providecommand\ldb{\llbracket}
\providecommand\rdb{\rrbracket}
\newcommand\pval{\mathsf{v}_p}
\newcommand{\gp}{\text{gp}}
\newcommand{\qf}{\text{qf}}
\newcommand{\supp}{\textsf{supp}}
\newcommand{\ii}{\mathcal{Irr}}
\newcommand{\uu}{\mathcal{U}}

\newcommand{\rad}{\text{Rad}}

%\keywords{BFD, bounded factorization domain, FFD, finite factorization domain, factorization, D+M construction, HFD, ACCP, atomic domain, monoid domain}
%
%\subjclass[2010]{Primary: 13A05, 13F15; Secondary: 13A15, 13G05}

\begin{document}
	
\mbox{}
\title{CMI 2026: Lecture Notes \\ \phantom{blank} \\ Intro to Commutative Monoids and Semirings}

\author{Mentor: Dr. Felix Gotti}


\address{Department of Mathematics\\MIT\\Cambridge, MA 02139}
\email{fgotti@mit.edu}


\bigskip
\maketitle
\fi



%%%%%%%%%%%%%%%%%%
\section{Commutative Monoids }


%%%%%%%%%%%%%%%%%%%%
%%%%%%%%%%%%%%%%%%%%
%%%%%%%%%%%%%%%%%%%%
\section*{Lecture 1 (by Harold)}

\subsection{General Notation.} In what follows (and throughout our weekly meetings), we let $\pp$, $\nn$, and $\nn_0$ denote the sets of primes, positive integers, and nonnegative integers, respectively. As it is customary, we let $\zz$, $\qq$, and $\rr$ denote the set of integers, rational numbers, and real numbers, respectively. In addition, for any $b,c \in \zz$, we let $\ldb b,c \rdb$ denote the discrete interval from $b$ to $c$:
\[
	\ldb b,c \rdb := \{n \in \zz : b \le n \le c\}.
\]
If $S$ is a subset of $\rr$ and $r \in \rr$, then we set $S_{\ge r} := \{s \in S : s \ge r\}$. We also use the notation $S_{\le r}$, $S_{> r}$, and $S_{< r}$ in a similar manner. 

%\medskip
%\subsection{Equivalence and Order Relations}
%
%TODO...


\medskip
%%%%%%%%%%%%%%%%%%%%%%%%%%%%%%
\subsection{Commutative Monoids and Abelian Groups}

A \emph{binary operation} on a nonempty set $M$ is a function $* \colon M \times M \to M$ (for any $a,b \in M$, we often write $a * b$ instead of $*(a,b)$). A pair $(M,*)$, where $M$ is a nonempty set and $*$ is a binary operation on $M$ is called a \emph{monoid} if the following two conditions hold:
\begin{enumerate}
	\item $a*(b*c) = (a*b)*c$ for all $a,b,c \in M$, and
	\smallskip
	
	\item there exists $e \in M$ such that $e*a = a*e = a$ for all $a \in M$.
\end{enumerate}
Let $(M,*)$ be a monoid, which we simply denote by $M$. Condition (1) is called \emph{associativity}. Condition~(2) can only be satisfied by one element $e$: if for some $e' \in M$ the equalities $e'*a = a*e' = a$ hold for all $a \in M$, then $e' = e$ (why?). The element $e$ is called the \emph{identity element} of the monoid $M$. An element $u \in M$ is called \emph{invertible} or a \emph{unit} if $u*v = v*u = e$ for some $v \in M$, in which case such an element $v$ is unique (why?) and $v$ is called the \emph{inverse} of $u$. Observe that $e*e = e$ and, therefore, $e$ is invertible. If $e$ is the only invertible element of $M$, then we say that $M$ is \emph{reduced}. %From this point on, we will denote a monoid $(M,*)$ simply by $M$ when the operation is either a generic operation $*$ or it is clear from the context. 
The set consisting of all invertible elements of $M$ is denoted by $\mathcal{U}(M)$ and called the \emph{group of units} of $M$. We say that $M$ is a \emph{group} if $M = \uu(M)$. It is not hard to verify that $\mathcal{U}(M)$ is always a group. If $a*b = b*a$ for all $a,b \in M$, then we say that $M$ is \emph{commutative}. A group that is a commutative (as a monoid) is called an \emph{abelian group}. From now on, we will tacitly assume that every monoid (resp., group) we refer to or deal with during the PRIMES reading period is commutative (resp., abelian). 
\smallskip

Although the definition of a monoid may seem a little bit abstract \emph{a priori}, we have actually dealt with many concrete and simple examples of monoids since we learned our first arithmetic operations.

\begin{example} \hfill
	The set $(\nn_0,+)$ of nonnegative integers is a commutative monoid under the standard operation of addition. However, it is not a group. Similarly, the the pairs $(\qq_{\ge 0},+)$ and $(\rr_{\ge 0},+)$ consisting of all nonnegative rational numbers and nonnegative real numbers, respectively, are monoids under the standard addition (but they are not groups). From now on, we simply write $\nn_0$, $\qq_{\ge 0 }$, and $\rr_{\ge 0}$ to refer to the additive monoids $(\nn_0,+)$, $(\qq_{\ge 0},+)$, and $(\rr_{\ge 0},+)$, respectively. 
\end{example}

\begin{example}
    The set $(\nn, \cdot)$ consisting of all positive integers under the standard multiplication is also a monoid, which we will denote simply by $\nn$. Observe that $\nn$ is not a group. \hfill $\square$
\end{example}
		
\begin{example}
    The set of integers $\zz$ under the standard addition is an abelian group and so a monoid. Similarly, the set $\qq$ of rationals and the set $\rr$ of reals are groups under the standard addition. \hfill $\blacksquare$
\end{example}



%%%%%%%%%%%%%%%%%%%%%%%%%%%%%%%%
\subsection{Monoid Homomorphisms} 

The most relevant maps between monoids are those that respect the corresponding binary operations, and so we need some terminology to refer to those maps.

\begin{definition}
	Let $M$ and $N$ be two monoids, and let $e_M$ and $e_N$ denote their respective identity elements. A function $f \colon M \to N$ is called a \emph{monoid homomorphism} if the following holds: 	for all $a,b \in M$,
	\[
		f(e_M) = e_N \quad \text{ and } f(a*b) = f(a)*f(b).
	\]
\end{definition}

 If $f \colon M \to N$ is a monoid homomorphism, then it is clear that $f(M)$ is a submonoid of $N$. We say that a monoid homomorphism is a \emph{monoid isomorphism} or simply an \emph{isomorphism} if it is a bijective function. Finally, we say that two monoids $M_1$ and $M_2$ are \emph{isomorphic} if there exists an isomorphism between them, in which case we write $M_1 \cong M_2$. The relation of being isomorphic is an equivalence relation in the class consisting of all monoids. A monoid homomorphism (resp., isomorphism) $f \colon M \to N$ is called a \emph{group homomorphism} (resp., \emph{group isomorphism}) if both~$M$ and~$N$ are groups.

%Finally, we say that two monoids are isomorphic if there exists an isomorphism between them. The relation of being isomorphic is an equivalence relation in the class consisting of all monoids.

\begin{example}
	Let $M$ be the monoid $\rr_{\ge 0}$, and let $N$ be the monoid $(\rr_{>0}, \cdot)$, where $\cdot$ denotes the standard multiplication of real numbers. Now consider the exponential function $f \colon M \to N$, that is, $f(r) = e^r$ for all $r \in M$. Since $f(0) = 1$ and $f(r+s) = f(r)f(s)$ for all $r,s \in \rr_{\ge 0}$, the function $f$ is indeed a monoid homomorphism. On the other hand, $f$ is not an isomorphism because $f$ is not surjective: in fact, the image of $f$ is the proper submonoid $S := \rr_{\ge 1}$ of $N$. However, if we slightly modify $f$ by restricting its co-domain to its image, then we obtain the monoid isomorphism $f \colon M \to S$. Hence the additive monoid $\rr_{\ge 0}$ and the multiplicative monoid $(\rr_{\ge 1}, \cdot)$ are isomorphic. \hfill $\blacksquare$
\end{example}




%%%%%%%%%%%%%%%%%%%%%%%%%%%%%%%
%%%%%%%%%%%%%%%%%%%%%%%%%%%%%%%
%%%%%%%%%%%%%%%%%%%%%%%%%%%%%%%
\section*{Lecture 2 (by Victor)}


\smallskip
%%%%%%%%%%%%%%%%%%%%%%%
\subsection{Submonoids}

For the rest of this section, let $M$ be a monoid with binary operation $*$ and identity element~$e$.

\begin{definition}
	A subset $S$ of $M$ is called a \emph{submonoid} of $M$ if $S$ contains~$e$ and is closed under the operation of $M$, which means that $a*b \in S$ for all $a,b \in S$.
\end{definition}

A submonoid $S$ of $M$ is called a \emph{subgroup} of $M$ if $S$ is a group. If $S$ is a submonoid (resp., a subgroup) of $M$ such that $S \neq M$, then $S$ is called a \emph{proper} submonoid (resp., subgroup) of $M$.

\begin{example}
    Observe that $\{e\}$ is a submonoid of $M$: the submonoid $\{e\}$ is clearly the smallest (under inclusion) submonoid of $M$. It is called the \emph{trivial submonoid} of $M$. On the other hand, we can see that the group of units $\uu(M)$ is a submonoid of $M$. \hfill $\blacksquare$
\end{example}

\begin{example} \label{ex:positive monoids}
	A submonoid $N$ of $\nn_0$ is called a \emph{numerical monoid} if the set $\nn_0 \setminus N$ is finite. For instance, for each $k \in \nn$, the set $N_k := \{0\} \cup \zz_{\ge k}$ is a numerical monoid under the standard operation of addition. \hfill $\blacksquare$
\end{example}
		
\begin{example}
    A submonoid of $\qq_{\ge 0}$ is called a \emph{Puiseux monoid}. For instance, for any positive $r \in \rr$, the sets $\{0\} \cup \qq_{> r}$ and $\{0\} \cup \qq_{\ge r}$ under the standard addition are Puiseux monoids that are not numerical monoids (prove this!). In addition, the set consisting of all nonnegative dyadic rationals is also a Puiseux monoid under the standard addition, which is not a numerical monoid. \hfill $\blacksquare$
\end{example}
		
\begin{example}
    A submonoid of $\rr_{\ge 0}$ is called a \emph{positive monoid}. For instance, the set
	\[
		\nn_0 + \nn_0 \pi = \{m + n \pi : m,n \in \nn_0\}
	\]
	is a positive monoid under the standard addition, which is not a numerical Puiseux monoid (prove this!) \hfill $\blacksquare$
\end{example}

One can readily check that the intersection of any arbitrary family of submonoids (resp., subgroups) of $M$ is a submonoid (resp., a subgroup) of $M$ (check this!). It turns out that the image of a monoid (resp., a group) under a monoid homomorphism is again a monoid (resp., a group).

\begin{proposition}
	Let $f \colon M \to N$ be a monoid homomorphism. Then the following statements hold.
	\begin{enumerate}
		\item $f(M)$ is a submonoid of $N$.
		\smallskip
		
		\item If $M$ is a group, then $f(M)$ is a group.
		% \smallskip
		%
		% \item $\emph{ker} f := \{m \in M : f(m) = e_N\}$ is a submonoid of $M$. 
	\end{enumerate}
\end{proposition}

\begin{proof}
	Exercise.
\end{proof}

Next we introduce a relevant class of submonoids, which are submonoids that are closed under divisibility. First, let us bring the notion of divisibility to the setting of commutative monoids.

\begin{definition} \label{def:divisibility}
	Let $M$ be a monoid, and let $a,b \in M$. 
	\begin{itemize}
		\item $b$ \emph{divides} $a$ if there exists $c \in M$ such that $a = bc$, in which case we write $b \mid_M a$.
		\smallskip
		
		\item $a$ and $b$ are \emph{associate} in $M$ if $a \mid_M b$ and $b \mid_M a$, in whch case we write $a \sim b$.
	\end{itemize}
\end{definition}

%Note that we can think of $M$ as an infinite directed graph $\text{Div}(M)$ whose vertices are the elements of $M$ and, for any $a,b \in M$, there is an edge from $a$ to $b$ if and only if $a \mid_M b$.
%\medskip

\noindent \textbf{Notation.} When the monoid $M$ is either $(\nn, \cdot)$ or $(\zz \setminus \{0\}, \cdot)$ under the standard multiplication, we write $a \mid b$ instead of $a \mid_M b$ provided that $a$ divides $b$ for some $a,b \in M$.
\medskip

As in the Definition~\ref{def:divisibility}, let $a$ and $b$ be elements of $M$. If $b \mid_M a$, then we say that $b$ is a \emph{divisor} of $a$ or that $a$ is \emph{divisible} by $b$. Observe that the divisibility relation $\mid_M$ is a binary relation on~$M$ that is reflexive and transitive. However, it is not an equivalence relation as there may be distinct elements $a,b \in M$ with $a \mid_M b$ and $b \mid_M a$. However, the divisibility relation on $M$ induces an order relation on the monoid consisting of all the associate classes of $M$. For $a \in M$, the set
\[
	aM^\times := \{au : u \in M^\times\}
\]
is called the \emph{associate class} of $a$ in $M$. It turns out that the set $M/M^\times$ consisting of all the associate classes of $M$ is a monoid under the operation $(aM^\times)(bM^\times) := abM^\times$. The monoid $M/M^\times$, also denoted by $M_{\text{red}}$, is called the \emph{reduced monoid} of $M$. Observe that the reduced monoid of $M$ has trivial group of units. We say that a monoid is \emph{reduced} if its group of units is trivial.

We are in a position to introduce divisor-closed submonoids.

\begin{definition}
    Let $M$ be a monoid, and let $S$ be a submonoid of $M$. We say that $S$ is a \emph{divisor-closed} submonoid of $M$ if for every pair $(m,s) \in M \times S$, the divisibility relation $m \mid_M s$ implies that $m \in S$. 
\end{definition}

In other words, a submonoid $S$ of $M$ is divisor-closed if all the elements of $M$ dividing at least an element of $S$ are precisely those in $S$.

\begin{example}
    Given a monoid $(M,*)$ with identity $e$, we have mentioned earlier that the trivial submonoid and the group of invertible elements $\uu(M)$ of $M$ is a submonoid of $M$. As every invertible element $u \in \uu(M)$ divides the identity $e$, we see that the trivial submonoid $\{e\}$ is divisor-closed if and only if $M$ is reduced. Moreover, $\uu(M)$ is always a divisor-closed submonoid of $M$. \hfill $\blacksquare$
\end{example}

\begin{example}
    Consider the positive monoid $M := \nn_0 + \nn_0\pi := \{a + b\pi : a,b \in \nn_0 \}$. It is clear that $\nn_0$ is a submonoid of $M$. Suppose that $d \mid_M s$ for some $(d,s) \in M \times \nn_0$, and let us show that $d \in \nn_0$. Take $a_1, a, b \in \nn_0$ such that $d := a+b\pi \mid_M a_1$. Then we can take $a_0, b_0 \in \nn_0$ such that $(a + b\pi) + (a_1 + b_1\pi) = a_0$ or, equivalently,
    \begin{equation*} %\label{eq:aux 20}
        (b+b_1)\pi = a_0 - a - a_1.
    \end{equation*}
    In this case $b+b_1 = 0$ as otherwise $\pi = \frac{a_0 - a - a_1}{b + b_1}$ would be rational (of course, we know this is not the case). From $b + b_1 \ge 0$ we obtain that $b = b_1 = 0$. Hence $d = a_1 \in \nn_0$, as desired. \hfill $\blacksquare$
\end{example}

It turns out that the only divisor closed submonoid of a Puiseux monoid are the trivial submonoid and the whole monoid.

\begin{proposition}
    Let $M$ be a Puiseux monoid. If $S$ is a divisor-closed submonoid, then $S$ is either $\{0\}$ or $M$.
\end{proposition}

\begin{proof}
    Let $S$ be a divisor-closed submonoid of the Puiseux monoid $M$. If $S = \{0\}$, we are done. Otherwise, fix $s \in S \setminus \{0\}$. Note that, for any nonzero $m \in M$, the equality $\mathsf{n}(m)\mathsf{d}(s)s = \mathsf{d}(m)\mathsf{n}(s)m$ guaranties that $m \mid_M as \in S$, whence $m \in S$. Hence $S=M$.
\end{proof}


\medskip
%%%%%%%%%%%%%%%%%%%
\subsection{Cancellative Monoids}

We proceed to introduce cancellative monoids. Let $M$ be a commutative monoid. An element $m \in M$ is called a \emph{cancellative element} if for all $a,b \in M$ the equality $ma = mb$ implies that $a=b$. The monoid $M$ is called a \emph{cancellative monoid} if every element of $M$ is cancellative.

Every group is obviously an example of a cancellative monoid, and most of the examples of monoids we have discussed so far are cancellative monoids that are not groups. For instance, every nonzero Puiseux/positive monoid is a cancellative monoid, and the multiplicative monoid $\mathbb{N}$ is a cancellative monoid.

\begin{example}
	The set of integers $\mathbb{Z}$ is a monoid under the standard multiplication, but as $0$ is not a cancellative element in $\mathbb{Z}$, the monoid $(\mathbb{Z}, \cdot)$ is not cancellative.
\end{example}

The property of being cancellative is preserved under isomorphisms.

\begin{proposition}
	Let $f \colon M \to M'$ be a monoid isomorphism. If $M$ is cancellative, then $f(M)$ is also cancellative.
\end{proposition}

\begin{proof}
	It suffices to prove that $f$ maps cancellative elements to cancellative elements. For this, take a cancellative element $m \in M$ and suppose that the equality  $f(m)a' = f(m)b'$ holds for some $a',b' \in M'$. As $f$ is surejective, we can take $a,b \in M$ with $f(a) = a'$ and $f(b) = b'$. Now observe that
	\[
		f(ma) = f(m)f(a) = f(m)a' = f(m)b' = f(m)f(b) = f(mb).
	\]
	As $f(ma) = f(mb)$, we can use the fact that $f$ is injective to deduce that $ma = mb$. Finally, the fact that $m$ is a cancellative element in $M'$ guarantees that $a=b$. Hence we conclude that $f$ preserves cancellative elements, which concludes our proof.
\end{proof}


Although the monoids we are mostly concerned in the scope of our research project are cancellative, it will be educative to take a brief look at the notion of power monoid, which are monoids that almost never are cancellative. Consider the following binary operation on the power set of $M$: for subsets $S$ and $T$ of the monoid $M$, set
\[
	ST := \{st : (s,t) \in S \times T\}.
\]
Such product induces a monoid structure on the power set of $M$, which is often called the \emph{large power monoid} of $M$. The restriction of such a product to the subset of $2^M$ consisting of all nonempty finite subset of $M$ is called the \emph{power monoid} of $M$.

\begin{example}
	The underlying set of the power monoid $\mathcal{P}_{\text{fin}}(\mathbb{N}_0)$ of the additive monoid $\mathbb{N}_0$ is the set of all nonempty finite set consisting of nonnegative integers. We can observe that the monoid $\mathcal{P}_{\text{fin}}(\mathbb{N}_0)$ is not cancellative as, for instance, 
	\[
		\{0,1\} + \{0,2\} = \{0,1,2,3\} = \{0,1\} + \{0,1,2\}
	\]
	even though $\{0,2\}$ and $\{0,1,2\}$ are distinct elements of $\mathcal{P}_{\text{fin}}(\mathbb{N}_0)$.
\end{example}

Every unit of a monoid must be a cancellative element and, because of this, every submonoid of a group must be cancellative. It turns out that every cancellative monoid is isomorphic to a submonoid of a group. This and more will be clear once we introduce the notion of the Grothendieck group of a given monoid.
%, which is a relevant construction we will see during our first meeting.






\bigskip
%%%%%%%%%%%%%%%%%%%%
%%%%%%%%%%%%%%%%%%%%
%%%%%%%%%%%%%%%%%%%%
\section*{Lecture 3 (by Harold)}





\medskip
%%%%%%%%%%%%%%%%%%%%%%%%%
\subsection{Localization}

In this section we discuss a powerful algebraic construction known as localization.

Let $M$ be a commutative monoid written multiplicatively, and let $S$ be a submonoid of $M$. Consider the binary relation $\sim$ on $M \times S$ defined as follows: for any $(m_1,s_1), (m_2,s_2) \in M \times S$, write
\[
	(m_1,s_1) \sim (m_2,s_2)
\]
if there exists $t \in S$ such that $t s_2 m_1 = ts_1 m_2$. One can readily verify that $\sim$ is an equivalence relation on the set $M \times S$. Let $S^{-1}M$ denote the set of equivalence classes determined by $\sim$ and, for any $(m,s) \in M \times S$, we let $\frac{m}s$ denote the equivalence class of $(m,s)$ in $S^{-1}M$. Now we can define a binary operation on the set $S^{-1}M$ in a natural way using the binary operation of $M$:
\[
	\frac{m_1}{s_1} \cdot \frac{m_2}{s_2} = \frac{m_1m_2}{s_1s_2}
\]
for all $(m_1,s_1), (m_2,s_2) \in M \times S$. It is routine to verify that the operation just defined on $S^{-1}M$ is actually well defined and turn $S^{-1}M$ into a monoid with identity element $\frac11$.

\begin{definition} \label{def:localization}
	Let $M$ be a commutative monoid, and let $S$ be a submonoid of $M$. The monoid $S^{-1}M$, also denoted by $M_S$ is called the \emph{localization} of $M$ at $S$.
\end{definition}

If $M$ is written additively, then the localization of $M$ at a submonoid $M$ is denoted by either $-S+M$ or $M_S$. It turns out that localizations of monoids satisfy the following universal property.

\begin{proposition} \label{prop:localization of monoid}
	\uses{def:localization}
	For a monoid $M$ and a submonoid $S$ of $M$, let $\iota \colon M \to M_S$ be the monoid homomorphism defined as follows: $\iota(m) = \frac{m}{1}$ for all $m \in M$. Then the following statements hold.
	\begin{enumerate}
		\item $\iota(S) \subseteq M_S^\times$.
		\smallskip
		
		\item If $f \colon M \to N$ is a monoid homomorphism such that $f(S) \subseteq N^\times$, then there exists a unique monoid homomorphism $g \colon M_S \to N$ such that $g \circ \phi = f$.
	\end{enumerate}
\end{proposition}

\begin{proof}
	(1) For any $s \in S$, we see that $\frac1s \in M_S$ and so we can immediately obtain that the inverse of $\iota(s)$ in $M_S$ is $\frac1s$ from the fact that $\iota(s) \cdot \frac1s = \frac{s}1 \cdot \frac1s = \frac11$. Thus, $\iota(S) \subseteq M_S^\times$.

	(2) Let now $f \colon M \to N$ be a monoid homomorphism such that $f(S) \subseteq N^\times$. Define $g \colon M_S \to N$ as follows: $g\big( \frac{m}{s}\big) = f(m) f(s)^{-1}$ for all $(m,s) \in M \times S$. It is routine to argue that $g$ is a well-defined monoid homomorphism. In addition, we see that
	\[
		g(\iota(m)) = g\Big( \frac{m}1 \Big) = f(m)f(1)^{-1} = f(m)
	\]
	for all $m \in M$, whence $g \circ \iota = f$. Finally, let us argue the uniqueness of~$g$. To do so, assume that $g' \colon M_S \to N$ is a monoid homomorphism % with $g'(S) \subseteq N^\times$ 
	such that $g' \circ \phi = f$. In this case, we obtain that
	\[
		g'\Big(\frac{m}s\Big) = g'(\iota(m)) \cdot g'(\iota(s)^{-1}) = f(m) \cdot f(s)^{-1} = g\Big(\frac{m}s\Big),
		%g'\Big(\frac{m}s\Big) = g'\Big(\frac{m}1\Big) g'\Big(\frac1s\Big) = g'(\iota(m)) \cdot g'(\iota(s)^{-1}) = f(m) \cdot f(s)^{-1} = g\Big(\frac{m}s\Big),
	\]
	for all $(m,s) \in M \times S$. Therefore the uniqueness of the monoid homomorphism $g$ follows, which concludes our proof.
\end{proof}

Let us take a look at a couple of examples.

\begin{example}
    Consider the multiplicative monoid $\nn$, and fix a prime $p \in \pp$. Note that the set $S := \{p^k : k \in \nn_0\}$ is a submonoid of $\nn_0$. Then we can localize $\nn$ at $S$ to obtain
    \[
        \nn_S := \Big\{ \frac{n}{2^k} : n,k \in \nn \Big\}.
    \]
    In particular, the localization of $\nn$ at $S$ is the multiplicative monoid consisting of all positive $p$-adic rationals. \hfill $\blacksquare$
\end{example}

\begin{example}
    We can now consider the same multiplicative monoid $\nn$ but localize it instead at its largest submonoid $S$ under inclusion, itself: $S = \nn$. In this case, we obtain that
    \[
        \nn_\nn := \Big\{ \frac{n}{d} : n,d \in \nn \Big\} = \qq_{>0}.
    \]
    Therefore the localization of $\nn$ at itself is the abelian group $\qq_{>0}$. \hfill $\blacksquare$
\end{example}

The localization of a monoid at itself is one of the most relevant examples of localization as it allows us to embed a given monoid into an abelian group provided that the monoid is cancellative. We will delve into this type of localization in the next section.



\medskip
%%%%%%%%%%%%%%%%%%%%%%%%%%%%%%%%%%%
\subsection{The Grothendieck Group}

It turns out that every cancellative monoid is isomorphic to a submonoid of an abelian group. Before proving this, let us introduce the notion a Grothendieck group.

\begin{definition} \label{def:Grothendieck group}
	Let $M$ be a commutative monoid. The localization of $M$ at~$M$ is denoted by $\gp(M)$ and called the \emph{Grothendieck group} of~$M$.
\end{definition}

%Let $\iota \colon M \to \mathcal{G}(M)$ be the localization homomorphism of $M$ with respect to~$M$, and let $G$ be an abelian group containing an isomorphic copy of $M$ as a submonoid. Let $f \colon M \to G$ be an injective monoid homomorphism. As $G$ is a group, Proposition~\ref{prop:localization of monoid} ensures the existence of a group homomorphism $g \colon \gp(M) \to G$ such that $g \circ \iota = f$. Observe that for all $m,s \in M$ with $g\big(\frac{m}s \big) = 1$, 
%\[
%	1 = g\Big( \frac{m}s\Big) = g\Big( \frac{m}1  \cdot \Big( \frac{s}1\Big)^{-1} \Big) = g\Big(\frac{m}1 \Big) g\Big( \Big( \frac{s}1\Big)^{-1} \Big) = g( \iota(m)) g(\iota(s)^{-1}) = f(m)f(s)^{-1},
%\]
%and so $f(s) = f(m)$. Thus, from the fact that $f$ is injective one obtains that $\ker g$ is trivial, which means that $g$ is also injective. Hence $G$ also contains an isomorphic copy of $\gp(M)$, namely, $g(\gp(M))$.

\begin{proposition} \label{prop:Grothendieck group, unversal property}
	\uses{prop:localization of monoid,def:Grothendieck group}
	Let $M$ be a cancellative monoid with Grothendieck group $\gp(M)$. Then any abelian group $G$ containing an isomorphic copy of~$M$ as a submonoid also contains an isomorphic copy of~$\gp(M)$ as a subgroup.
\end{proposition}

\begin{proof}
	Let $\iota \colon M \to \mathcal{G}(M)$ be the localization homomorphism of $M$ with respect to~$M$, and let $G$ be an abelian group containing an isomorphic copy of $M$ as a submonoid. Then there is a monoid homomorphism $f \colon M \to G$ that is injective. As $G$ is a group, Proposition~\ref{prop:localization of monoid} ensures the existence of a group homomorphism $g \colon \gp(M) \to G$ such that $g \circ \iota = f$. Observe that for all $m,s \in M$ with $g\big(\frac{m}s \big) = 1$, 
	\[
		1 \! = \! g\Big( \frac{m}s\Big) \! = \! g\Big( \frac{m}1  \Big( \frac{s}1\Big)^{-1} \Big) \! = \! g\Big(\frac{m}1 \Big) g\Big( \Big( \frac{s}1\Big)^{-1} \Big) \! = \! g( \iota(m)) g(\iota(s)^{-1}) \! = \! f(m)f(s)^{-1},
	\]
	and so $f(s) = f(m)$. Thus, from the fact that $f$ is injective one obtains that $\ker g$ is trivial, which means that $g$ is also injective. Hence $G$ also contains an isomorphic copy of $\gp(M)$, namely, $g(\gp(M))$.
\end{proof}

Here is a useful result.

\begin{proposition}
	Let $A$ be an abelian group and let $M$ be a submonoid of $A$. Then there exists a subgroup $G$ of $A$ such that $M \subseteq G \subseteq A$ and $G$ is isomorphic to the Grothendieck group of $M$.
\end{proposition}

\begin{proof}
	Exercise.
\end{proof}

\begin{theorem} \label{thm:isomorphism extends to gp}
	\uses{prop:localization of monoid,def:Grothendieck group}
	Let $f \colon M \to N$ be an isomorphism of monoids, and let $i_M \colon M \to \gp(M)$ and $i_N \colon N \to \gp(N)$ be the canonical embeddings of $M$ and $N$ into their corresponding Grothendieck groups. 
%	Then the following statements hold.
%	\begin{enumerate}
%		\item 
	There is a unique group homomorphism $F \colon \gp(M) \to \gp(N)$ extending $f$, which means that $F \circ i_M = i_N \circ f$ or that the following diagram commutes:
	\begin{equation} \label{eq:extension homomorphism to Grothendieck groups}
		\begin{tikzcd}
			M \arrow[r, "f"] \arrow[d, "i_M"'] & N \arrow[d, "i_N"] \\
			\gp(M) \arrow[r, dashed, "F"'] & \gp(N)
		\end{tikzcd}
		%\caption{A group homomorphism $F$ extending a given monoid homomorphism $f$.}
		%\label{fig:extension homomorphism to Grothendieck groups}
	\end{equation}
		%\item 
		
		\noindent Moreover, if $f$ is injective (resp., surjective, bijective) so is group homomorphism~$F$.
	%\end{enumerate}
\end{theorem}

\begin{proof}
	As $\gp(N)$ is a group, $i_N \circ f$ sends every element of $M$ to a unit, and so there exists a group homomorphism $F \colon \gp(M) \to \gp(N)$ such that the diagram~\ref{eq:extension homomorphism to Grothendieck groups} commutes, that is,
	\[
		F \circ i_M = i_N \circ f,
	\]
	and so the first statement of the theorem is established.
	\smallskip
	
	For the last statement of the theorem, it suffices to address separately the cases when $f$ is surjective and injective, which we proceed to do.
	\begin{itemize}
		\item Assume first that $f$ is a surjective monoid homomorphism. Then there is a monoid homomorphism $g \colon N \to M$ such that $f \circ g = \mathrm{id}_N$. It follows that:
		\[
			F \circ i_M \circ g = (F \circ i_M) \circ g = i_N \circ f \circ g = i_N.
		\]
		where $F \circ i_M = i_N \circ f$ (by the part of the theorem we have already established). Therefore $F \circ G = \mathrm{id}_{\mathsf{G}(N)}$, which implies that $F \circ G \colon \gp(N) \to \gp(N)$ is a monoid homomorphism making the diagram~\eqref{eq:extension homomorphism to Grothendieck groups} commute. As a result, $F \circ G = \mathrm{id}_{\mathsf{G}(N)}$, and so $F$ is surjective.
		\smallskip
		
		\item Now assume that $f$ is injective. Then we can take a surjective monoid homomorphism $g \colon N \to M$, and so there exists a group homomorphism $G \colon \gp(N) \to \gp(M)$ such that $G \circ i_N = i_M \circ g$. Then we can proceed similarly to argue that $G \circ F = \mathrm{id}_{\gp(M)}$. Hence $F$ is injective.
	\end{itemize}
\end{proof}

It is clear that every abelian group is its own Grothendieck group. Let us take a look at some further example of localization of monoids and Grothendieck groups.

\begin{example}
	The Grothendieck group of the additive monoid $\nn_0$ is $-\nn_0 + \nn_0$, which is the abelian group~$\zz$. Similarly, we can check that the Grothendieck groups of $\qq_{\ge 0}$ and $\rr_{\ge 0}$ are $\qq$ and $\rr$, respectively. \hfill $\blacksquare$
\end{example}

\begin{example}
	 We proceed to determine all possible localizations of the additive monoid~$\nn_0$. Fix a nontrivial submonoid $S$ of $\nn_0$, and let us show that $-S + \nn_0 = \zz$. We only need to verify that $\zz \subseteq -S + \nn_0$ as it is clear that $-S + \nn_0 \subseteq \zz$. As $S$ is nontrivial, $S$ must contain a positive integer~$s$. Now fix $n \in \zz$. If $n \ge 0$, then $n \in 0 + \nn_0 \subseteq -S + \nn_0$. Otherwise, we can take $m \in \nn$ large enough so that $-ms \le n$, and so $-ms \in -S$ and $n+ms \in \nn_0$, from which we obtain that $n = -cs + (n + cs) \in -S + \nn_0$. Hence the inclusion $-S + \nn_0$ also holds. Thus, when we localize $\nn_0$ at any nontrivial submonoid we obtain~$\zz$. In particular, $\zz$ is the Grothendieck group of $\nn_0$. \hfill $\blacksquare$
\end{example}



\begin{example} \label{ex:Grothendieck group of a NM}
	\uses{def:Grothendieck group}
	Let $N$ be a numerical monoid. Then $\mathcal{G}(N) = \{b-c : b,c \in N\} \subseteq \zz$. As $N$ is co-finite, it must contain two consecutive positive integers and so $1 \in \mathcal{G}(N)$. Thus, $-1 \in \mathcal{G}(N)$ and, as a consequence $\zz = \nn \cdot 1 + \nn \cdot (-1) \subseteq \mathcal{G}(N)$. Hence $\gp(N) = \zz$. \hfill $\blacksquare$
%	\smallskip
%	Clearly, the inclusion map $\iota \colon N \to \zz$ is an injective monoid homomorphism. Let us argue that we can take the codomain of $\iota$ as the Grothendieck group of $N$. In order to do so, it suffices to check that and let us argue that the inclusion map $\iota \colon N \to \zz$ $\zeta \colon N \subseteq \nn_0 \to \zz$ be the inclusion homomorphism. and let us argue that the Grothendieck group $G := \gp(M)$ of $N$ can be taken to be $\zz$. To do so, take....
\end{example}

\begin{example}
	Let $q$ be a positive rational number, and let $M_q$ denote the Puiseux monoid $\nn_0[q]$, which can be generated by the set $\{q^n : n \in \nn_0\}$. Let us argue that the Grothendieck group
	\[
		G := \{r-s : r,s \in M_q\}
	\]
	of $M_q$ is $\zz[q]$. Since any element of $M_q$ has the form $f(q)$ for some $f \in \nn_0[x]$, any element of $G$ has the form $f(q) - g(q) = (f-g)(q)$ for some $f,g \in \nn_0[x]$ and so $f(q) - g(q) = (f-g)(q) \in \zz[q]$ because $f-g \in \zz[x]$. Hence the inclusion $G \subseteq \zz[q]$ holds. %implies that $f(q)$  $M_q$ are\zz[q]$ is an abelian group  The inclusion $G \subseteq \zz[q]$ clearly holds because any... Now observe that
%	\[
%		\mathcal{G}(M_q) := \big\{ f(q) - g(q) : f,g \in \nn_0[q] \big\} \subseteq \zz[q].
%	\]
	For the reverse inclusion, note that $1 \in M_q \subseteq G$, which guarantees that $-1 \in G$. Thus, $\zz \subseteq G$. Now write $q = a/b$ for some $a,b \in \nn$ such that $\gcd(a,b) = 1$. For each $k \in \nn$, the fact that $\gcd(a^{k+1},b) = 1$ allows us to take coefficients $c,d \in \zz$ such that $1 = c a^{k+1} + d b$, which implies that
	\begin{equation} \label{eq:Bezout}
		\frac1{b^{k+1}} = c q^{k+1} + d \frac1{b^k}.
	\end{equation}
	For each $k \in \nn_0$, it follows from~\eqref{eq:Bezout} that $\frac1{b^{k+1}} \in G$ provided that $\frac1{b^k} \in G$. Thus, it follows from induction on $k$ that $\frac1{b^k} \in G$ for every $k \in \nn_0$ (the base case, $1 \in G$, has been already established at the beginning of the paragraph). As $G$ is a group, the inclusion $\big\{ \pm \frac1{b^k} : k \in \nn_0 \big\} \subseteq G$ holds, and so $\zz[q] = \{\pm q^k : k\in \nn_0 \} \subseteq G$. As a consequence, $\gp(\nn_0[q]) = \zz[q]$. \hfill $\blacksquare$
%	As the inclusion $\phi \colon M_q = \nn_0[q] \to \zz[q]$ is an injective monoid homomorphism, it must factor through the universal injective monoid homomorphism $\iota \colon M_q \mapsto \gp(M_q)$, which means that there exists a group homomorphism $f \colon \gp(M_q) \to \zz[q]$ such that $\phi(r) = f(\iota(r))$ for all $r \in M$.
\end{example}





\bigskip
%%%%%%%%%%%%%%%%%%%%%%%%%%%%%%
%%%%%%%%%%%%%%%%%%%%%%%%%%%%%%
%%%%%%%%%%%%%%%%%%%%%%%%%%%%%%
\section*{Lecture 4 (by Victor)}
\medskip


\medskip
%%%%%%%%%%%%%%%%%%%%%%%%%%%%%%%%%%%%%%%%%
\subsection{Sums and Products of Monoids}

Let $I$ be a nonempty set. For each $i \in I$, let $S_i$ be a set. The \emph{product} of the sets $S_i$ is the set $\prod_{i \in I} S_i$ consisting of all functions $f \colon I \to \bigcup_{i \in I} S_i$ with $f(i) \in S_i$ for all $i \in I$:
\[
	\prod_{i \in I} S_i := \Big\{ f \colon I \to \bigcup_{i \in I} S_i : f(i) \in S_i \text{ for all } i \in I \Big\}.
\]
We often denote a function $f \in \prod_{i \in I} S_i$ as $(c_i)_{i \in I}$, where $c_i := f(i) \in M_i$ for all $i \in I$. If $S_i = S$ for all $i \in I$, we write $S^I$ instead of $\prod_{i \in I} S_i$. Observe that if $S_j$ is empty for some $j \in I$, then $\prod_{i \in I} S_i$ is also empty. Also, if $S_i$ is nonempty for all $i \in I$ but $S_j$ is an infinite set for some $j \in I$, then the product $\prod_{i \in I} S_i$ is an infinite set. If $I$ is finite and $S_i$ is a finite set for every $i \in I$, then
\[
	\Big{|} \prod_{i \in I} S_i\Big{|} = \prod_{i \in I} |S_i|.
\]
For $n \in \nn$ and sets $S_1, \dots, S_n$, we can also consider the set of $n$-tuples
\[
	S_1 \times \dots \times S_n := \{(s_1, \dots, s_n) : s_i \in S_i \text{ for all } i \in \ldb 1,n \rdb\},
\]
and we write $S^n$ instead of $S_1 \times \dots \times S_n$ if $S_i = S$ for every $i \in \ldb 1, n \rdb$. Although $S_1 \times \dots \times S_n$ and $\prod_{i \in \ldb 1,n \rdb} S_i$ are technically different (as in the second one the order does not matter), they are essentially the same set because the only intrinsic attribute of a set is its cardinality and
\[
	|S_1 \times \dots \times S_n| = |S_{\sigma(1)} \times \dots \times S_{\sigma(n)}| = \bigg{|} \prod_{i \in \ldb 1,n \rdb} S_i \bigg{|}
\]
for any permutation $\sigma$ of $\ldb1,n\rdb$. Motivated by this, when we write the functions in $\prod_{i \in S} S_i$ as $(c_i)_{i \in I}$, we call them $I$-\emph{tuples} or, simply, \emph{tuples}. The product of sets that are also monoids can be naturally turned into a monoid. 
\smallskip

For the rest of this section, let $I$ be a nonempty set, and let $(M_i)_{i \in I}$ be a family of monoids, where we let $1_{M_i}$ denote the identity element of the monoid $M_i$. Then set
\[
	M := \prod_{i \in I} M_i.
\]
We can define a binary operation on $M$ in the following natural way:
\begin{equation} \label{eq:componentwise operation}
	(b_i)_{i \in I} \cdot (c_i)_{i \in I} := (b_ic_i)_{i \in I}
\end{equation}
for all $I$-tuples $(b_i)_{i \in I}$ and $(c_i)_{i \in I}$ in $M$. It is routine to show that, under the binary operation we have just defined, which we call the \emph{componentwise operation}, $M$ is a monoid with identity element $(1_{M_i})_{i \in I}$.

\begin{definition}
	For a nonempty familiy of monoids $(M_i)_{i \in I}$, the product set $\prod_{i \in I} M_i$ under componentwise operation is called the \emph{product monoid} of the family $(M_i)_{i \in I}$.
\end{definition}

Observe that the group of units of $M$ is the product of the groups of units of the componentwise monoid: $M^\times = \prod_{i \in I} M_i^\times$, and this immediately implies both statements of the following proposition.

\begin{proposition}
	For a nonempty set $I$, let $(M_i)_{i \in I}$ be a family of monoids. Then the following statements hold.
	\begin{enumerate}
		\item $M^\times = \prod_{i \in I} M_i^\times$.
		\smallskip
		
		\item $M$ is reduced if and only if $M_i$ is reduced for all $i \in I$.
		\smallskip
		
		\item $M$ is a group if and only if $M_i$ is a group for all $i \in I$.
	\end{enumerate}
\end{proposition}

An $I$-tuple $(b_i)_{i \in I} \in \prod_{i \in I} M_i$ is said to have \emph{finite support} if $b_i$ is the identity element of $M_i$ for all but finitely many $i \in I$. We let $\bigoplus_{i \in I} M_i$ denote the subset of the product monoid $\prod_{i \in I} M_i$ consisting of all the $I$-tuples having finite support. Clearly, $\bigoplus_{i \in I} M_i$ is a submonoid of $\prod_{i \in I} M_i$.

\begin{definition}
	For a nonempty set $I$, let $(M_i)_{i \in I}$ be a family of monoids. The submonoid $\bigoplus_{i \in I} M_i$ of $\prod_{i \in I} M_i$ is called the \emph{sum} or \emph{internal product} of the family $(M_i)_{i \in I}$. 
\end{definition}

If $M_i = M$ for all $i \in I$, the internal product of the family $(M_i)_{i \in I}$ is denoted by $M^{(I)}$ instead of $\bigoplus_{i \in I} M$.
\color{black}



\bigskip
\section*{Exercises -- Intro to Commutative Monoids}

\begin{exe}
    Let $\nn_0[x]$ be the set consisting of all polynomials with nonnegative integer coefficients. For $q \in \qq_{>0}$, consider the following subset of $\qq$:
	\[
		\nn_0[q] := \{p(q) : p(x) \in \nn_0[x]\}.
	\]
	\begin{enumerate}
        \item What is the group of units of the additive monoid $(\nn_0[q],+)$?
        \smallskip
        
		\item Find the group of unit of the multiplicative monoid $(\nn_0[q] \setminus \{0\}, \cdot)$.
    \end{enumerate}
\end{exe}
\smallskip

\begin{exe}
	Let $f \colon M \to N$ be a monoid homomorphism. Then the following statements hold.
	\begin{enumerate}
		\item $f(M)$ is a submonoid of $N$.
		\smallskip
		
		\item If $M$ is a group, then $f(M)$ is a group.
		% \smallskip
		
		% \item $\text{ker} f := \{m \in M : f(m) = e_N\}$ is a submonoid of $M$. 
	\end{enumerate}
\end{exe}
\smallskip

\begin{exe}
	\hfill
	\begin{enumerate}
		\item Prove that every submonoid of $\nn_0$ is isomorphic to a numerical monoid.
		\smallskip
		
		\item Prove that, for each $r \in \rr_{> 0}$, the set $\{0\} \cup \qq_{> r}$ is a Puiseux monoid that is not isomorphic to any numerical monoid.
		\smallskip
		
		\item Prove that the set $\nn_0 + \nn_0 \pi$ is a positive monoid that is not isomorphic to any Puiseux monoid.
	\end{enumerate}
\end{exe}
\smallskip

\begin{exe}
    Let $M$ be a monoid. Prove that the arbitrary intersection of submonoids of $M$ is again a submonoid of $M$.
\end{exe}

\begin{exe}
    \hfill
    \begin{enumerate}
        \item Find all divisor-closed submonoids of the positive monoid
        \[
            \{a + \pi b : a,b \in \nn_0 \}.
        \]
        \smallskip
        
        \item Prove that $\nn_0$ is a divisor-closed submonoid of
        \[
            M := \{a + b\sqrt{2} : a,b \in \nn_0 \}.
        \]
        \smallskip
        
        \item Prove that $\nn$ a submonoid of the multiplicative monoid
        \[
            \{a + b\sqrt{2} : a,b \in \nn_0 \text{ and } a+b \ge 1 \}
        \]
        that is not a divisor-closed submonoid.
    \end{enumerate}
\end{exe}
\smallskip

\begin{exe}
	Let $M$ be a cancellative monoid. Prove that if $M$ has only finitely many elements, then~$M$ is a group.
\end{exe}
\smallskip


\begin{exe}
    Find all possible commutative monoid such that the power monoid $\mathcal{P}_{\text{fin}}(M)$ is cancellative monoid.
\end{exe}
\smallskip


\begin{exe}
    Find all possible localizations of the multiplicative monoid $\nn$.
\end{exe}
\smallskip

\begin{exe}
	 \hfill
	\begin{enumerate}
			\item[(a)] For each $r \in \rr_{> 0}$, find the Grothendieck group of the Puiseux monoid $\{0\} \cup \qq_{> r}$.
			\smallskip
			
			\item[(b)] Prove that the Grothendieck group of the positive monoid $\nn_0 + \nn_0 \pi$ is isomorphic to the abelian group $\zz \times \zz$ (under component-wise addition).
		\end{enumerate}
\end{exe}
\smallskip

\begin{exe}
	Find the Grothendieck group of the additive monoid $(\nn_0[q],+)$.
\end{exe}
\smallskip

\begin{exe}
	Find the Grothendieck group of the multiplicative monoid $(\nn_0[q] \setminus \{0\}, \cdot)$.
\end{exe}
\smallskip

\section*{Appendix: Homework \& Research Transition Problem Set}

\subsection*{HW1}
TBA

\subsection*{HW2}
TBA

\subsection*{HW3} Let $\mathbb{N}_0[x]$ be the set consisting of all polynomials with nonnegative integer coefficients, that is

\[
    \mathbb{N}_0[x] := \left\{ \sum_{i=o}^n c_ix^i : n \in \mathbb{N}_0 \text{ and } c_0, \ldots, c_n \in \mathbb{N}_0 \right\}.
\]

For each $r \in \mathbb{R}$, let
\[
    \mathbb{N}_0[r] := \{ p(r) : p(x) \in \mathbb{N}_0[x] \}.
\]


\medskip
\noindent\textbf{Ex 1.}

\begin{enumerate}
  \item[(a)] Prove that, for any $r \in \mathbb{R}$, the pair $(\mathbb{N}_0[r], +)$ is a monoid if ``$+$'' is the standard addition in $\mathbb{R}$.
  \item[(b)] For which $q \in \mathbb{Q}$ is $(\mathbb{N}_0[q], +)$ a group?
  \item[(c)] For which $r \in \mathbb{R}$ is $(\mathbb{N}_0[r], +)$ isomorphic to the additive monoid $\mathbb{N}_0[x]$ (under the standard addition of polynomials)?
\end{enumerate}

\medskip
\noindent\textbf{Ex 2.} Fix $r \in \mathbb{Q}_{>0}$, and consider the Puiseux monoid $(\mathbb{N}_0[r], +)$. 


\begin{enumerate}
  \item[(a)] Prove that for any positive $q \in \mathbb{N}_0[r]$ we can write
  \begin{equation}\label{eq:I}
    q = c_0(q) + \sum_{n \in \mathbb{N}} c_n(q) r^n, \tag{I}
  \end{equation}
  
  where $(c_n(q))_{n \ge 0}$ is a sequence of nonnegative integers such that there are only finitely many $n \in \mathbb{N}_0$ with $c_n(q) \neq 0$.

  \item[(b)] Prove that there exists a unique sum decomposition \eqref{eq:I} such that
  \[
    c_n(q) < d(r) \qquad \text{for all } n \ge 1.
  \]
  This is called the \emph{CSD} of $q$ in $\mathbb{N}_0[r]$.

  \item[(c)] For any $q_1, q_2 \in \mathbb{N}_0[r]$, 
  prove that $q_1 \mid_{\mathbb{N}_0[r]} q_2$ if and only if \textbf{3FDPM}.
\end{enumerate}




\bigskip



\ifdefined\ISBLUEPRINT\else
\end{document}
\fi
